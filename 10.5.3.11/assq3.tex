\let\negmedspace\undefined
\let\negthickspace\undefined
\documentclass[journal,12pt,onecolumn]{IEEEtran}
\usepackage{cite}
\usepackage{amsmath,amssymb,amsfonts,amsthm}
\usepackage{algorithmic}
\usepackage{graphicx}
\usepackage{textcomp}
\usepackage{xcolor}
\usepackage{txfonts}
\usepackage{listings}
\usepackage{enumitem}
\usepackage{mathtools}
\usepackage{gensymb}

\usepackage{tkz-euclide} % loads  TikZ and tkz-base
\usepackage{listings}



\newtheorem{theorem}{Theorem}[section]
\newtheorem{problem}{Problem}
\newtheorem{proposition}{Proposition}[section]
\newtheorem{lemma}{Lemma}[section]
\newtheorem{corollary}[theorem]{Corollary}
\newtheorem{example}{Example}[section]
\newtheorem{definition}[problem]{Definition}
%\newtheorem{thm}{Theorem}[section] 
%\newtheorem{defn}[thm]{Definition}
%\newtheorem{algorithm}{Algorithm}[section]
%\newtheorem{cor}{Corollary}
\newcommand{\BEQA}{\begin{eqnarray}}
\newcommand{\EEQA}{\end{eqnarray}}
\newcommand{\system}[1]{\stackrel{#1}{\rightarrow}}

\newcommand{\define}{\stackrel{\triangle}{=}}
\theoremstyle{remark}
\newtheorem{rem}{Remark}
%\bibliographystyle{ieeetr}
\begin{document}
%
\providecommand{\pr}[1]{\ensuremath{\Pr\left(#1\right)}}
\providecommand{\prt}[2]{\ensuremath{p_{#1}^{\left(#2\right)} }}        % own macro for this question
\providecommand{\qfunc}[1]{\ensuremath{Q\left(#1\right)}}
\providecommand{\sbrak}[1]{\ensuremath{{}\left[#1\right]}}
\providecommand{\lsbrak}[1]{\ensuremath{{}\left[#1\right.}}
\providecommand{\rsbrak}[1]{\ensuremath{{}\left.#1\right]}}
\providecommand{\brak}[1]{\ensuremath{\left(#1\right)}}
\providecommand{\lbrak}[1]{\ensuremath{\left(#1\right.}}
\providecommand{\rbrak}[1]{\ensuremath{\left.#1\right)}}
\providecommand{\cbrak}[1]{\ensuremath{\left\{#1\right\}}}
\providecommand{\lcbrak}[1]{\ensuremath{\left\{#1\right.}}
\providecommand{\rcbrak}[1]{\ensuremath{\left.#1\right\}}}
\newcommand{\sgn}{\mathop{\mathrm{sgn}}}
\providecommand{\abs}[1]{\left\vert#1\right\vert}
\providecommand{\res}[1]{\Res\displaylimits_{#1}} 
\providecommand{\norm}[1]{\left\lVert#1\right\rVert}
%\providecommand{\norm}[1]{\lVert#1\rVert}
\providecommand{\mtx}[1]{\mathbf{#1}}
\providecommand{\mean}[1]{E\left[ #1 \right]}
\providecommand{\cond}[2]{#1\middle|#2}
\providecommand{\fourier}{\overset{\mathcal{F}}{ \rightleftharpoons}}
\newenvironment{amatrix}[1]{%
  \left(\begin{array}{@{}*{#1}{c}|c@{}}
}{%
  \end{array}\right)
}
%\providecommand{\hilbert}{\overset{\mathcal{H}}{ \rightleftharpoons}}
%\providecommand{\system}{\overset{\mathcal{H}}{ \longleftrightarrow}}
	%\newcommand{\solution}[2]{\textbf{Solution:}{#1}}
\newcommand{\solution}{\noindent \textbf{Solution: }}
\newcommand{\cosec}{\,\text{cosec}\,}
\providecommand{\dec}[2]{\ensuremath{\overset{#1}{\underset{#2}{\gtrless}}}}
\newcommand{\myvec}[1]{\ensuremath{\begin{pmatrix}#1\end{pmatrix}}}
\newcommand{\mydet}[1]{\ensuremath{\begin{vmatrix}#1\end{vmatrix}}}
\newcommand{\myaugvec}[2]{\ensuremath{\begin{amatrix}{#1}#2\end{amatrix}}}
\providecommand{\rank}{\text{rank}}
\providecommand{\pr}[1]{\ensuremath{\Pr\left(#1\right)}}
\providecommand{\qfunc}[1]{\ensuremath{Q\left(#1\right)}}
	\newcommand*{\permcomb}[4][0mu]{{{}^{#3}\mkern#1#2_{#4}}}
\newcommand*{\perm}[1][-3mu]{\permcomb[#1]{P}}
\newcommand*{\comb}[1][-1mu]{\permcomb[#1]{C}}
\providecommand{\qfunc}[1]{\ensuremath{Q\left(#1\right)}}
\providecommand{\gauss}[2]{\mathcal{N}\ensuremath{\left(#1,#2\right)}}
\providecommand{\diff}[2]{\ensuremath{\frac{d{#1}}{d{#2}}}}
\providecommand{\myceil}[1]{\left \lceil #1 \right \rceil }
\newcommand\figref{Fig.~\ref}
\newcommand\tabref{Table~\ref}
\newcommand{\sinc}{\,\text{sinc}\,}
\newcommand{\rect}{\,\text{rect}\,}
%%
%	%\newcommand{\solution}[2]{\textbf{Solution:}{#1}}
%\newcommand{\solution}{\noindent \textbf{Solution: }}
%\newcommand{\cosec}{\,\text{cosec}\,}
%\numberwithin{equation}{section}
%\numberwithin{equation}{subsection}
%\numberwithin{problem}{section}
%\numberwithin{definition}{section}
%\makeatletter
%\@addtoreset{figure}{problem}
%\makeatother

%\let\StandardTheFigure\thefigure
\let\vec\mathbf

\bibliographystyle{IEEEtran}





\bigskip

\renewcommand{\thefigure}{\theenumi}
\renewcommand{\thetable}{\theenumi}
%\renewcommand{\theequation}{\theenumi}


\title{Discrete Assignment}
\author{Shravya Kantayapalam\\ EE23BTECH11030}
\maketitle
\textbf{Question} \ 10.5.3.11\
% Define variables for the arithmetic progression

 If the sum of the first \( n \) terms of an AP is \( 4n - n^2 \), what is the first term (\( S_1 \))? What is the sum of the first two terms? What is the second term? Similarly, find the 3rd, the 10th, and the \( n \)th terms.
 
\textbf{Answer}

\begin{document}

If the sum of the first $n$ terms of an AP is $4n - n^2$.

Let $S_n$ represent the sum of the first $n$ terms of the arithmetic progression (AP).

\begin{align*}
S_n &= 4n - n^2
\end{align*}


\[
S_n = \frac{n}{2}[2a + (n-1)d]
\]

Where:
\begin{align*}
S_n & \text{ is the sum of the first } n \text{ terms,} \\
a & \text{ is the first term,} \\
d & \text{ is the common difference.}
\end{align*}


\[
4n - n^2 = \frac{n}{2}[2a + (n-1)d]
\]

To find the first term \(a\):
\begin{align*}
4 - n &= a + (n-1)d \\
4 - n &= a + (n-1)d \\
4 - n &= a + nd - d \\
4 - n &= a - d + nd - d \\
4 - n &= a - d(1 - n)
\end{align*}

Comparing coefficients:
\[
a = 4, \quad d = -1
\]

Thus, the first term \(S_1\) is \(a = 4\).

To find the sum of the first two terms (\(S_2\)):
\begin{align*}
S_2 &= 2\left(\frac{2a + (n-1)d}{2}\right) \\
&= 2\left(\frac{2(4) + (2-1)(-1)}{2}\right) \\
&= 2\left(\frac{8 - 1}{2}\right) \\
&= 2\left(\frac{7}{2}\right) \\
&= 7
\end{align*}

To find the second term (\(S_2 - S_1\)):
\begin{align*}
S_2 - S_1 &= 7 - 4 \\
&= 3
\end{align*}

To find the values of \(a\) and \(d\) 

\[ S_n = 4n - n^2 \]

\[ S_n = \frac{n}{2}[2a + (n-1)d] \]
Where:
\begin{align*}
S_n & \text{ is the sum of the first } n \text{ terms,} \\
a & \text{ is the first term, and} \\
d & \text{ is the common difference.}
\end{align*}

\[ 4n - n^2 = \frac{n}{2}[2a + (n-1)d] \]


\textbf{To find \(a\), the first term:}
\begin{align*}
4n - n^2 &= n(a + (n-1)d) \\
4 - n &= a + (n-1)d
\end{align*}


\textbf{To find \(d\), the common difference:}
\[ d = a_2 - a_1 \]


\textbf{Solving for \(a\) and \(d\):}
Using the equation \(4 - n = a + (n-1)d\), we can derive \(a\) and \(d\).

When \(n = 1\):
\[ 4 - 1 = a + (1-1)d \]
\[ 3 = a \]

When \(n = 2\):
\[ 4 - 2 = a + (2-1)d \]
\[ 2 = a + d \]

Substitute \(a = 3\) into the equation \(2 = a + d\):
\[ 2 = 3 + d \]
\[ d = -1 \]

Thus, the first term \(a\) is 3 and the common difference \(d\) is -1.

\begin{itemize}
    \item First term (\(S_1\)): \(a = 3\)
    \item Sum of the first two terms (\(S_2\)): 
    \[S_2 = 2\left(\frac{2a + (n-1)d}{2}\right)\]
    \[S_2 = 2\left(\frac{2(3) + (2-1)(-1)}{2}\right) = 2(3) = 6\]
    \item Second term: \(S_2 - S_1 = 6 - 3 = 3\)
\end{itemize}

In general, the \(n\)th term is \(a + (n-1)d\), so:
\begin{itemize}
    \item The \(3\)rd term: \(3 + (3-1)(-1) = 3 - 2 = 1\)
    \item The \(10\)th term: \(3 + (10-1)(-1) = 3 - 9 = -6\)
    \item The \(n\)th term: \(3 + (n-1)(-1) = 3 - n + 1 = 4 - n\)
\end{itemize}

\end{document}


